\hspace{4.5mm}
\subsection{Introdução}
Este documento de requisitos descreve as funcionalidades e as características do sistema a ser desenvolvido para um restaurante, visando solucionar os problemas de controle de estimativa de entrega dos pratos e gerenciamento de pedidos. O objetivo é melhorar a eficiência operacional e a satisfação dos clientes.

\subsection{Escopo}
O sistema será desenvolvido para gerenciar o fluxo de pedidos, estimar o tempo de entrega e fornecer relatórios que auxiliem no controle de desempenho do Mangaba's Kitchen, sendo unica e exclusivamente direcionado ao mesmo. Ele será implantado para o ambiente de cozinha quanto para a equipe de salão do restaurante.

\subsection{Requisitos Funcionais}
Os requisitos funcionais especificam as funcionalidades que o sistema deve proporcionar para atender efetivamente às necessidades do cliente e resolver os problemas identificados. Esses requisitos detalham as operações e capacidades que o sistema deve possuir para garantir que todas as operações do restaurante sejam gerenciadas de forma eficiente e eficaz.

\begin{itemize}
    \item \textbf{Estimativa de Tempo de Entrega:} O sistema deve calcular o tempo estimado para a preparação e entrega de cada prato e exibi-lo ao cliente no momento da realização do pedido.
    
    \item \textbf{Controle de Pedidos:} O sistema deve permitir que a equipe do restaurante acompanhe o status de cada pedido, desde o momento em que ele é recebido até sua finalização (preparado e entregue).
    
    \item \textbf{Atualização do Status dos Pedidos:} O sistema deve permitir que a cozinha atualize o status dos pedidos, informando quando o prato está em preparação e quando está pronto para ser entregue.
    
    \item \textbf{Histórico de Pedidos:} O sistema deve armazenar um histórico de todos os pedidos realizados, permitindo a geração de relatórios sobre o tempo de preparo e entrega para análise posterior.
    
    \item \textbf{Notificações para os Garçons:} O sistema deve enviar notificações automáticas aos garçons informando sobre o status dos pedidos ativos na interface daquele garçom específico, especialmente quando o prato estiver pronto para ser entregue.
\end{itemize}

\subsection{Requisitos Não Funcionais}
Os requisitos não funcionais descrevem as características e qualidades que o sistema deve ter para garantir sua eficácia e boa operação. Eles incluem aspectos como desempenho, segurança, usabilidade e manutenção. No contexto do projeto para o Mangaba Kitchen, esses requisitos garantem que o sistema não só funcione corretamente, mas também atenda a padrões elevados de qualidade e eficiência.

\begin{itemize}
    \item \textbf{Performance:} O sistema deve calcular e exibir o tempo estimado de entrega em tempo real, sem atrasos significativos, para garantir a precisão das informações fornecidas aos clientes.
    
    \item \textbf{Confiabilidade:} O sistema deve ser altamente confiável, garantindo que os pedidos não sejam perdidos ou processados incorretamente, independentemente da quantidade de pedidos sendo feitos simultaneamente.
    
    \item \textbf{Usabilidade:} O sistema deve ser fácil de usar tanto para os funcionários do restaurante quanto para os clientes, com uma interface simples e intuitiva.
    
    \item \textbf{Escalabilidade:} O sistema deve ser capaz de lidar com um aumento no número de pedidos e pratos sem comprometer o desempenho ou a precisão das estimativas de tempo.
    
    \item \textbf{Segurança:} O sistema deve proteger os dados dos funcionários entre si, clientes e dos pedidos contra acessos não autorizados, garantindo a privacidade e a segurança das informações.
\end{itemize}

\subsection{Conclusão}
Este documento de requisitos especifica as funcionalidades e características que o sistema proposto deve apresentar para melhorar o controle de estimativa de entrega e o gerenciamento de pedidos no restaurante. A implementação desse sistema contribuirá para otimizar a operação interna e proporcionar uma melhor experiência aos clientes.
\hspace{4.5mm}
\subsection{Introdução}
O restaurante Mangaba Kitchen enfrenta desafios significativos relacionados ao controle de entrega de pratos e ao gerenciamento dos pedidos dos clientes. A falta de um sistema eficiente para prever o tempo de entrega e organizar os pedidos resulta em uma experiência insatisfatória para os clientes, além de afetar a eficiência e a precisão do serviço prestado. Para resolver esses problemas, é essencial implementar um sistema informatizado que possa otimizar a operação do restaurante. O objetivo principal é melhorar a estimativa de tempo de entrega e o controle de pedidos, proporcionando uma comunicação mais clara com os clientes e uma gestão mais eficaz da cozinha. Com isso, o restaurante poderá reduzir a frustração dos clientes, aumentar a eficiência operacional e aprimorar a qualidade geral do serviço. As soluções propostas visam automatizar o controle de pedidos, integrar estimativas de entrega em tempo real e gerar relatórios detalhados para promover melhorias contínuas, assegurando um ambiente de trabalho mais organizado e um atendimento mais satisfatório.

\subsection{Principais Objetivos}
O objetivo é propor uma solução que atenda às necessidades do restaurante em relação à estimativa de tempo de entrega dos pratos e ao controle de pedidos, resolvendo os seguintes problemas:

\begin{itemize}
    \item \textbf{Controle de acesso e segurança operacional:} Criar um sistema de login com senhas individuais para garantir segurança e responsabilidade dos funcionários no gerenciamento dos pedidos e operações. Esse controle de acesso personalizado permitirá rastrear ações individuais, identificando possíveis falhas e atribuindo responsabilidades. Além disso, o sistema assegurará que apenas funcionários autorizados tenham acesso a determinadas funcionalidades, aumentando a segurança dos dados e a eficiência nas operações diárias.
    
    \item \textbf{Melhoria na estimativa de tempo de entrega:} 
    Implementar um sistema que ofereça estimativas precisas e em tempo real, permitindo que o restaurante informe aos clientes o tempo exato de espera. Além de reduzir a incerteza e a frustração, essa funcionalidade também ajudará na organização interna, permitindo ajustes eficientes no fluxo de trabalho da cozinha. Isso resultará em uma experiência mais agradável para os clientes e em uma operação mais coordenada e ágil para a equipe do restaurante.
    
    \item \textbf{Automatização e otimização do controle de pedidos:} 
    Desenvolver uma plataforma centralizada que possibilite o acompanhamento de cada pedido em suas diferentes fases (recebido, em preparação, pronto). Isso permitirá uma visão em tempo real do status de cada pedido, facilitando a coordenação entre garçons e cozinheiros, além de reduzir o risco de erros ou atrasos. A plataforma também possibilitará uma gestão mais eficiente das filas de espera, garantindo que os pedidos sejam processados de maneira organizada e com maior precisão.

    \item \textbf{Redução de erros operacionais:} 
    Ao integrar o controle de pedidos com o sistema de estimativa de entrega, o restaurante poderá reduzir erros, como confusões na sequência de pedidos ou tempos de preparo imprecisos, resultando em maior eficiência e qualidade do serviço. Essa integração permitirá que a equipe acompanhe em tempo real o status de cada pedido, facilitando a coordenação entre salão e cozinha. Além disso, ajudará a evitar atrasos e melhorar a experiência do cliente, garantindo que os prazos de entrega sejam cumpridos de forma mais precisa e organizada.

    \item \textbf{Análise e melhoria contínua:} 
    Com a coleta de dados sobre o tempo de preparo e entrega dos pedidos, será possível gerar relatórios que auxiliem o restaurante a identificar áreas de melhoria, possibilitando ajustes contínuos nos processos de cozinha e atendimento ao cliente. Esses relatórios fornecerão insights valiosos sobre gargalos operacionais e desempenho da equipe, permitindo uma gestão mais informada e estratégica. Além disso, a análise dos dados ajudará a otimizar o tempo de produção e entrega, garantindo um serviço mais ágil e satisfatório para os clientes.
\end{itemize}

A solução proposta visa, portanto, melhorar a experiência dos clientes, aumentar a eficiência operacional do restaurante e fornecer dados precisos para análises futuras.
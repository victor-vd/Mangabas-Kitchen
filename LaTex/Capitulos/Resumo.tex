\hspace{4.5mm}
O projeto de informatização do Mangaba's Kitchen visa desenvolver um sistema informatizado para o restaurante Mangaba Kitchen, com o objetivo de otimizar o controle de pedidos e estimativas de entrega. A iniciativa busca resolver problemas relacionados à falta de precisão nas previsões de tempo e à dificuldade em gerenciar o fluxo de pedidos, impactando negativamente a experiência dos clientes e a eficiência operacional do restaurante. A solução proposta inclui a criação de um sistema integrado que melhora a comunicação entre garçons e cozinha, oferecendo estimativas de tempo mais precisas e um controle mais eficiente dos pedidos.
O sistema contará com três interfaces principais: uma para o gerente, que gerencia o cadastro de funcionários, pratos e formas de pagamento, além de gerar relatórios de comissão e faturamento; uma para os garçons, que permitirá o gerenciamento de mesas, adição de pedidos e visualização das estimativas de entrega; e uma para os cozinheiros, que possibilitará o gerenciamento dos pedidos em preparação, a visualização das receitas e o controle do tempo de preparo. Cada interface será projetada para atender às necessidades específicas dos usuários e garantir uma operação mais organizada.
Além das funcionalidades principais, como o controle de pedidos e estimativas de tempo, o sistema incluirá recursos de segurança, como logins individuais para os funcionários, e relatórios detalhados para melhorar a eficiência operacional. A informatização do pagamento servirá apenas para controle, sem transações financeiras diretas. Com essas melhorias, o projeto visa não apenas reduzir erros e atrasos, mas também proporcionar uma experiência mais satisfatória para os clientes e uma gestão interna mais eficiente.
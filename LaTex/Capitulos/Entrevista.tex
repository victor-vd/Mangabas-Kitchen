\hspace{4.5mm}
A entrevista com os clientes do restaurante Mangaba's Kitchen foi conduzida para esclarecer alguns pontos que estavam pendentes no projeto de automação do sistema de atendimento. O foco da conversa foi entender melhor as expectativas dos proprietários em relação à padronização do atendimento e à redução de prejuízos. A proposta da empresa foi desenvolver um sistema que utilizasse tablets, tanto para os funcionários quanto para os clientes, oferecendo funcionalidades que incluíssem desde a visualização dos pratos até a identificação dos responsáveis pela preparação. A ideia central era criar uma solução que atendesse às demandas do restaurante de forma eficiente e ágil, garantindo um melhor controle dos processos internos e uma experiência mais organizada para os clientes.
\par
Para os garçons, o sistema incluirá uma tela de cadastro, associando cada funcionário ao nome e função. Além disso, será possível vincular cada garçom às mesas que ele atende, registrando o valor dos pratos vendidos e as especificações dos pedidos feitos pelos clientes. No final de cada dia, o sistema gerará um resumo com as informações dos pratos preparados e as vendas realizadas por cada garçom. Já para os cozinheiros, a proposta é semelhante, com o sistema registrando os pratos preparados, o cozinheiro responsável por cada pedido e um relatório diário de desempenho. Essa automação permitirá um maior controle da produtividade e uma melhor coordenação entre o salão e a cozinha.
\par
Além das funcionalidades voltadas para a equipe, o sistema incluirá tablets em cada mesa, permitindo que os clientes visualizem o cardápio, valores dos pratos e especificações, assim como o nome do garçom responsável pelo atendimento. Na cozinha, uma televisão exibirá o nome dos pratos, o tempo estimado de preparo e as especificações dos pedidos, facilitando o trabalho dos cozinheiros e garantindo uma entrega mais precisa e dentro do prazo. Embora os clientes não tenham especificado detalhes de layout, eles demonstraram satisfação com a logo criada pela equipe de design e prometeram fornecer referências para ajudar na criação de uma interface que atenda melhor às suas preferências estéticas.
\hspace{4.5mm}
\subsection{Introdução} O restaurante cliente enfrenta desafios críticos no controle da entrega de pratos e no gerenciamento dos pedidos dos clientes. A falta de um sistema eficiente para estimar com precisão o tempo de entrega e organizar adequadamente os pedidos compromete a experiência dos clientes, prejudica a eficiência operacional e afeta a precisão do serviço. Esses problemas geram dificuldades na comunicação do tempo de espera, aumento da frustração dos clientes e falta de controle sobre o fluxo de trabalho da cozinha, impactando negativamente tanto a satisfação do cliente quanto a eficácia do restaurante.

\subsection{Descrição do Problema} O problema pode ser subdividido em dois aspectos principais:

\textbf{Estimativa de Tempo de Entrega} Atualmente, o restaurante não possui um sistema eficiente para prever o tempo necessário para preparar e entregar os pratos aos clientes. Isso resulta em: 
\begin{itemize} 
    \item Dificuldade em informar o tempo exato de entrega ao cliente; 
    \item Incerteza em relação ao prazo, causando frustração para os clientes; 
    \item Falta de controle sobre o fluxo de trabalho da cozinha, o que pode gerar atrasos ou confusão na preparação dos pratos. 
\end{itemize}

\textbf{Controle dos Pedidos} O sistema de controle de pedidos também apresenta dificuldades em: 
\begin{itemize} 
    \item Monitorar os pedidos em andamento, identificando quais estão prontos e quais ainda precisam ser preparados; 
    \item Gerenciar o fluxo de trabalho de maneira eficiente para otimizar a produção e evitar acúmulo de pedidos; 
    \item Manter um histórico preciso dos pedidos para gerar relatórios de desempenho e melhorar a operação futura. 
\end{itemize}

\textbf{Impacto do Problema} Esses problemas resultam em uma série de desafios para o restaurante, incluindo: 
\begin{itemize} 
    \item Aumento da insatisfação dos clientes devido à demora e à falta de comunicação precisa sobre os tempos de entrega; 
    \item Ineficiência operacional que pode resultar em desperdício de tempo e recursos; 
    \item Dificuldade em melhorar os processos internos devido à falta de dados precisos sobre o desempenho dos pedidos. 
\end{itemize}

\textbf{Proposta de Solução} Para mitigar os problemas identificados, sugerimos a implementação de um sistema que ofereça: 
\begin{itemize} 
    \item Estimativas precisas e em tempo real do tempo de entrega de cada prato; 
    \item Um painel de controle que permite ao restaurante visualizar todos os pedidos em andamento, juntamente com seu status e prazo estimado de conclusão; 
    \item Relatórios de desempenho que permitam analisar tempos de preparo e entrega para melhorias contínuas. 
\end{itemize}

\subsection{Conclusão} A implementação de um sistema integrado para controle de estimativa de tempo e pedidos ajudará o restaurante a melhorar sua eficiência, reduzir atrasos e oferecer uma experiência mais satisfatória aos seus clientes.
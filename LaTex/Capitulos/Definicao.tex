\hspace{4.5mm}
O restaurante Mangaba's Kitchen, que oferece almoço à tarde e, à noite, pizza e hambúrgueres, está em processo de informatização de seu sistema de pedidos. O objetivo principal, segundo o cliente, é otimizar o controle sobre a estimativa de entrega dos pratos e melhorar a gestão dos pedidos. O restaurante deseja implantar um sistema com login e senha individual para cada funcionário, inclusive cozinheiros, com renovação obrigatória da senha a cada três meses. Além disso, o app precisa se integrar aos pedidos feitos por delivery e fornecer uma interface para os garçons registrarem os pedidos feitos localmente, facilitando a comunicação entre o salão e a cozinha.
\par
A funcionalidade principal para o cozinheiro/chef, considerando o que o cliente falou, é a tela que permite adicionar os pedidos à "fila de espera", ajustando automaticamente o tempo estimado de preparo. O cozinheiro também pode ajustar manualmente o tempo de preparo dos pratos conforme necessário. Para auxiliar na execução, cada prato na fila deve exibir a receita correspondente. Para os garçons, a interface possibilitará a adição dos pedidos de cada mesa, com um sistema que calcula o valor total conforme os itens são inseridos. No final do dia, o sistema poderá gerar relatórios de faturamento, separando por mesa e por garçom.
\par
Além disso, foi reforçado que o sistema precisa incluir funcionalidades para gerar relatórios de comissão dos garçons e aplicar descontos financeiros em caso de erros na preparação dos pratos, descontando o valor dos responsáveis (cozinha ou garçom, dependendo da falha). O sistema também deve permitir a informatização dos métodos de pagamento, registrando se o pagamento foi feito à vista, por crédito ou débito. Embora haja uma integração com o delivery, o sistema não é um serviço de entrega terceirizado como o iFood, sendo exclusivo para uso interno do restaurante.
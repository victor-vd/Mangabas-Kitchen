\hspace{4.5mm}
\subsection {Viabilidade Organizacional e Operacional}
A viabilidade organizacional e operacional do projeto de informatização do Mangaba's Kitchen é altamente favorável, dada a forma como o sistema está alinhado com os objetivos estratégicos do restaurante e sua integração planejada nas operações diárias. O sistema, que inclui funcionalidades como controle de acesso e gerenciamento de pedidos, foi projetado para melhorar a eficiência e a comunicação interna entre gerentes, garçons e cozinheiros, facilitando o fluxo de trabalho e a coordenação entre as diferentes áreas. A escolha de dispositivos como iPads e uma TV touch, combinada com uma interface intuitiva, garante uma transição suave e a aceitação do sistema pela equipe. A abordagem metodológica Scrum permite um desenvolvimento iterativo, ajustando o sistema com base no feedback dos usuários e nas necessidades emergentes, o que assegura que o sistema continue a ser relevante e eficaz. Esse planejamento detalhado e a adaptação contínua asseguram que a implementação seja harmoniosa, sem interromper o fluxo de trabalho existente, e que a equipe possa rapidamente usufruir das melhorias proporcionadas pelo novo sistema.
\subsection {Viabilidade Econômica}
A viabilidade econômica do projeto de informatização do Mangaba's Kitchen está intrinsecamente ligada à análise das funcionalidades propostas e ao custo associado ao desenvolvimento do sistema. O sistema detalhado e bem delineado pelo contratante permite uma identificação clara dos requisitos, facilitando a realização de uma avaliação precisa e objetiva. A partir dessa análise, é possível planejar com maior acurácia os recursos necessários, assegurando que o desenvolvimento seja concluído nos prazos e custos estimados. Além disso, a clareza dos objetivos e a definição das funcionalidades essenciais contribuem para uma previsão de manutenção simplificada e eficiente, reduzindo possíveis complicações futuras.
\par
Considerando o esforço de desenvolvimento, que abrange a criação de interfaces específicas para gerentes, garçons e cozinheiros, além de funcionalidades como controle de acesso, gerenciamento de pedidos e visualização de receitas, conclui-se que o projeto é viável economicamente. A estrutura proposta não só se alinha com as necessidades operacionais do restaurante, mas também proporciona uma plataforma escalável, capaz de se adaptar a futuras expansões ou melhorias. A simplicidade na manutenção, derivada da organização e clareza do sistema, reforça ainda mais essa viabilidade, garantindo que o Mangaba Kitchen possa operar com eficiência e agilidade, maximizando o retorno sobre o investimento feito no desenvolvimento dessa solução tecnológica.
\subsection {Viabilidade Técnica}
A viabilidade técnica do projeto é fortemente sustentada pela clareza e simplicidade dos requisitos estabelecidos, reduzindo a complexidade do desenvolvimento e permitindo uma implementação eficiente. A contratada possui ampla experiência nas ferramentas e tecnologias necessárias para a criação de sistemas de gestão, o que garante uma execução precisa e alinhada às expectativas do cliente. Além disso, a escolha de dispositivos como iPads para os garçons e uma TV touch (sugerida pela Contratada) para os cozinheiros evidencia uma estratégia bem planejada que aproveita as capacidades de hardware já estabelecidas, facilitando a integração e operação do sistema no ambiente do restaurante.
\subsection {Competência da Equipe}
A competência da equipe envolvida no projeto, aliada ao conhecimento profundo das ferramentas utilizadas, assegura que todas as funcionalidades requisitadas serão entregues conforme o cronograma e as especificações. A contratada não apenas domina os aspectos técnicos necessários, mas também compreende as nuances do ambiente de um restaurante, permitindo que o sistema desenvolvido seja robusto, intuitivo e de fácil manutenção. Essa combinação de fatores reforça a confiança na execução do projeto, garantindo que ele será uma solução eficaz e duradoura para as necessidades operacionais do Mangaba's Kitchen.
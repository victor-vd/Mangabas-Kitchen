\hspace{4.5mm}
Portanto, a proposta de informatização do Mangaba Kitchen visa transformar como o restaurante gerencia suas operações diárias, abordando diretamente as dificuldades enfrentadas na coordenação entre a equipe e no controle de pedidos. A introdução de um sistema integrado que inclui interfaces específicas para gerentes, garçons e cozinheiros permitirá uma gestão mais eficiente dos pedidos, um controle aprimorado das estimativas de entrega e uma comunicação mais fluida entre as áreas. Com funcionalidades como cadastro e gerenciamento de funcionários, acompanhamento de pedidos e geração de relatórios financeiros, o sistema visa não apenas otimizar a operação interna, mas também melhorar a experiência do cliente, reduzindo atrasos e erros que impactam a satisfação dos consumidores.
\par
A viabilidade econômica e técnica do projeto está bem sustentada pela clareza dos requisitos e pela experiência da equipe de desenvolvimento, que possui conhecimento profundo das ferramentas e tecnologias necessárias. O cronograma estruturado utilizando o framework Scrum, com fases definidas e sprints regulares, assegura que cada etapa do desenvolvimento será cuidadosamente monitorada e ajustada conforme necessário. A combinação de uma estratégia bem planejada, a escolha adequada de dispositivos e a abordagem iterativa garantem que o sistema será implementado eficientemente, alinhando-se às necessidades operacionais do Mangaba Kitchen e promovendo uma operação mais ágil e eficiente.
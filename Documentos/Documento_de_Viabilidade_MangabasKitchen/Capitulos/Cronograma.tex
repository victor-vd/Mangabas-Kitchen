\hspace{4.5mm}
O cronograma do projeto de informatização do Mangaba's Kitchen foi estruturado utilizando o framework Scrum, que se destaca pela sua abordagem simples, iterativa e incremental. O projeto está dividido em três fases principais: especificação de requisitos, onde será feito o levantamento e análise dos requisitos e refinamento do backlog, projeto e protótipo, onde será definido as funcionalidades e especificações técnicas que serão necessárias para realizar o desenvolvimento do sistema, implementação, onde acontecerá o desenvolvimento do sistema. A fase de especificação de requisitos foi agendada para ser concluída até o fim do segundo sprint, dia 22 de setembro, com prazo total de um mês. Após essa fase, a equipe iniciará a fase de projeto, que está agendada para ser concluída até o terceiro sprint, dia 6 de outubro, com prazo total de duas semanas, seguida pela implementação do sistema, que se estenderá até o dia 5 de novembro. Esse planejamento detalhado garante que cada etapa do desenvolvimento seja cuidadosamente monitorada e ajustada conforme necessário.
\par
Os sprints, que têm a duração de duas semanas, começaram no dia 25 de agosto, com reuniões semanais de Scrum para revisar o progresso e ajustar as estratégias conforme necessário. Os sprints continuam regularmente, com o quarto sprint previsto para terminar no dia 5 de novembro, marcando o fim da fase de implementação do sistema. O cronograma inclui ainda um sprint adicional para refinamento, testes e ajustes, com o último realizado em 10 de novembro, seguido pela Scrum Meeting final, que ocorrerá no dia 13 de novembro, onde será feita uma revisão final do projeto, para ser apresentado para o cliente no dia 15 de novembro.
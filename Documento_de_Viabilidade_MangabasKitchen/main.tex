\documentclass[12pt,oneside,a4paper,article]{abntex2}
\usepackage[utf8]{inputenc} % Codificação do documento
\usepackage[T1]{fontenc}    % Seleção de código de fonte.
\usepackage[brazil]{babel}  % Idioma do documento
\usepackage{graphicx}       % Inclusão de gráficos
\usepackage{tabularx}       % Tabelas avançadas
\usepackage{amsmath}        % Melhorias em matemática
\usepackage{lipsum}         % Geração de texto dummy
\usepackage{authblk}
\usepackage{parskip}

% Configurações específicas do abntex2
% Aqui você pode adicionar configurações específicas, como redefinições de comandos
% ou adições de novos pacotes que são essenciais para o seu documento.

% Carrega o pacote abntex2cite para citações
\usepackage[alf]{abntex2cite} % ou use [num] para citações numéricas

\usepackage[left=3cm,right=2cm,top=3cm,bottom=2cm]{geometry} % Margens
\usepackage{setspace}       % Espaçamento entre linhas
% %\usepackage{natbib}         % Formatação de bibliografia

% Informações de título
\title{\textbf{Documento de Viabilidade de Mangaba's Kitchen}}
\author{Eve
llyn Costa \thanks{evellyn.pedroso@ucsal.edu.br}}
\author{Guilherme Vieira \thanks{guilhermenovais.vieira@ucsal.edu.br}}
\author{Henrique Silva \thanks{henrique.souza@ucsal.edu.br}}
\author[1]{Lucas Santos \thanks{lucassantos.garrido@ucsal.edu.br}}
\author[1]{Maria Luisa \thanks{marialuisacarvalho.silva@ucsal.edu.br}}
\author[1]{Victor Dantas \thanks{victor.dantas@ucsal.edu.br} }
\author[1]{Vinicius Kauan \thanks{viniciuskauan.cruz@ucsal.edu.br} }
\author[1*]{Elton Figueiredo \thanks{elton.figueiredo@pro.ucsal.br}}
\affil{
    Engenharia de Software \par
    Escola de Tecnologias \par
    Universidade Católica do Salvador (UCSAL) \par
    Av. Prof. Pinto de Aguiar, 2589 Pituaçu, CEP: 41740-090 \par
    Salvador/BA, Brasil
}
% \affil[1]{\textit {\{evellyn.pedroso,henrique.souza,lucassantos.garrido,guilhermenovais.vieira\}@ucsal.edu.br}}
% affil[1]{\textit {\{marialuisacarvalho.silva,victor.dantas,viniciuskauan.cruz\}@ucsal.edu.br}}
% \affil[1*]{\textit {\{elton.figueiredo\}@pro.ucsal.edu.br}}




\date{Agosto 2024}

\newpage

\ifthenelse{\equal{\ABNTEXisarticle}{true}}{%
\renewcommand{\maketitlehookb}{}
}{}

% Configurações de aparência do PDF final
% \usepackage{hyperref} % para inserir links
\hypersetup{
    colorlinks=false,       % false: boxed links; true: colored links
    pdfborder={0 0 0},      % remove as bordas ao redor dos links
}

\newpage
\renewcommand*{\Authsep}{, }
\renewcommand*{\Authand}{, }
\renewcommand*{\Authands}{, }
\renewcommand*{\Affilfont}{\normalsize\normalfont}
\renewcommand*{\Authfont}{\bfseries}    % make author names boldface    
\setlength{\affilsep}{2em}   % set the space between author and affiliation

\newsavebox\affbox



\newpage


\begin{document}

\begin{center}
    \includegraphics[width=0.3\textwidth]{imagens-template/ucsal_logo.png} 
\end{center}
{\let\newpage\relax\maketitle}

\newpage
\begin{resumoumacoluna}
\hspace{4.5mm}
O Mangaba Kitchen, um restaurante que oferece almoço durante a tarde e pizza e hambúrgueres à noite, enfrenta desafios na gestão de pedidos e no controle das estimativas de entrega. Para melhorar a eficiência operacional e a experiência do cliente, o restaurante propõe desenvolver um sistema integrado que incluirá interfaces específicas para gerentes, garçons e cozinheiros. O sistema permitirá o cadastro e gerenciamento de funcionários, a entrada e acompanhamento dos pedidos, e a geração de relatórios financeiros e de comissões. Além disso, o sistema auxiliará no controle do tempo de preparo e na visualização das receitas, proporcionando uma solução completa para otimizar o atendimento e a coordenação interna.
\vspace{\onelineskip}
 
\noindent
\textbf{Palavras-chaves}: Scrum, Sprint, Viabilidade.
\end{resumoumacoluna}

\textual

\vspace{12mm}

\chapter{Introdução}
    \vspace{-6mm}
    \hspace{4.5mm}
O Mangaba Kitchen é um restaurante que se destaca por seu cardápio variado, com refeições completas durante o almoço e especialidades em pizza e hambúrgueres à noite. Com o crescimento das operações e a complexidade crescente no gerenciamento de pedidos, o restaurante identificou a necessidade de uma solução tecnológica para melhorar a eficiência e a precisão no atendimento. Atualmente, a coordenação entre os garçons e a cozinha enfrenta desafios que podem levar a atrasos e erros nos pedidos, impactando diretamente a satisfação dos clientes.
\vspace{12mm}

\newpage
\chapter{Objetivo}
    \vspace{-6mm}
    \hspace{4.5mm}
\subsection{Introdução}
O restaurante Mangaba Kitchen enfrenta desafios significativos relacionados ao controle de entrega de pratos e ao gerenciamento dos pedidos dos clientes. A falta de um sistema eficiente para prever o tempo de entrega e organizar os pedidos resulta em uma experiência insatisfatória para os clientes, além de afetar a eficiência e a precisão do serviço prestado. Para resolver esses problemas, é essencial implementar um sistema informatizado que possa otimizar a operação do restaurante. O objetivo principal é melhorar a estimativa de tempo de entrega e o controle de pedidos, proporcionando uma comunicação mais clara com os clientes e uma gestão mais eficaz da cozinha. Com isso, o restaurante poderá reduzir a frustração dos clientes, aumentar a eficiência operacional e aprimorar a qualidade geral do serviço. As soluções propostas visam automatizar o controle de pedidos, integrar estimativas de entrega em tempo real e gerar relatórios detalhados para promover melhorias contínuas, assegurando um ambiente de trabalho mais organizado e um atendimento mais satisfatório.

\subsection{Principais Objetivos}
O objetivo é propor uma solução que atenda às necessidades do restaurante em relação à estimativa de tempo de entrega dos pratos e ao controle de pedidos, resolvendo os seguintes problemas:

\begin{itemize}
    \item \textbf{Controle de acesso e segurança operacional:} Criar um sistema de login com senhas individuais para garantir segurança e responsabilidade dos funcionários no gerenciamento dos pedidos e operações. Esse controle de acesso personalizado permitirá rastrear ações individuais, identificando possíveis falhas e atribuindo responsabilidades. Além disso, o sistema assegurará que apenas funcionários autorizados tenham acesso a determinadas funcionalidades, aumentando a segurança dos dados e a eficiência nas operações diárias.
    
    \item \textbf{Melhoria na estimativa de tempo de entrega:} 
    Implementar um sistema que ofereça estimativas precisas e em tempo real, permitindo que o restaurante informe aos clientes o tempo exato de espera. Além de reduzir a incerteza e a frustração, essa funcionalidade também ajudará na organização interna, permitindo ajustes eficientes no fluxo de trabalho da cozinha. Isso resultará em uma experiência mais agradável para os clientes e em uma operação mais coordenada e ágil para a equipe do restaurante.
    
    \item \textbf{Automatização e otimização do controle de pedidos:} 
    Desenvolver uma plataforma centralizada que possibilite o acompanhamento de cada pedido em suas diferentes fases (recebido, em preparação, pronto). Isso permitirá uma visão em tempo real do status de cada pedido, facilitando a coordenação entre garçons e cozinheiros, além de reduzir o risco de erros ou atrasos. A plataforma também possibilitará uma gestão mais eficiente das filas de espera, garantindo que os pedidos sejam processados de maneira organizada e com maior precisão.

    \item \textbf{Redução de erros operacionais:} 
    Ao integrar o controle de pedidos com o sistema de estimativa de entrega, o restaurante poderá reduzir erros, como confusões na sequência de pedidos ou tempos de preparo imprecisos, resultando em maior eficiência e qualidade do serviço. Essa integração permitirá que a equipe acompanhe em tempo real o status de cada pedido, facilitando a coordenação entre salão e cozinha. Além disso, ajudará a evitar atrasos e melhorar a experiência do cliente, garantindo que os prazos de entrega sejam cumpridos de forma mais precisa e organizada.

    \item \textbf{Análise e melhoria contínua:} 
    Com a coleta de dados sobre o tempo de preparo e entrega dos pedidos, será possível gerar relatórios que auxiliem o restaurante a identificar áreas de melhoria, possibilitando ajustes contínuos nos processos de cozinha e atendimento ao cliente. Esses relatórios fornecerão insights valiosos sobre gargalos operacionais e desempenho da equipe, permitindo uma gestão mais informada e estratégica. Além disso, a análise dos dados ajudará a otimizar o tempo de produção e entrega, garantindo um serviço mais ágil e satisfatório para os clientes.
\end{itemize}

A solução proposta visa, portanto, melhorar a experiência dos clientes, aumentar a eficiência operacional do restaurante e fornecer dados precisos para análises futuras.
\vspace{12mm}

\newpage
\chapter{Análise de Viabilidade}
    \vspace{-6mm}
    \input{Capitulos/Desenvolvimento}
\vspace{12mm}

\newpage
\chapter{Cronograma}
    \vspace{-6mm}
    \hspace{4.5mm}
A fase de especificação de requisitos foi agendada para ser concluída até o fim do quarto sprint, com prazo total de um mês, onde será produzido os seguintes documentos: problema, objetivo, documento de requisitos, documento de viabilidade, identidade visual e figma do projeto. Após essa fase, a equipe iniciará a fase de projeto, que está agendada para ser concluída até o terceiro sprint, dia 6 de outubro, com prazo total de duas semanas.
\par
Os sprints, que têm a duração de uma semana, eles começaram no dia 25 de agosto, com reuniões Scrum nos dias úteis, para revisar o progresso e ajustar as estratégias conforme necessário. Os sprints continuam regularmente, com o quarto sprint previsto para terminar no dia 15 de setembro, marcando o fim da fase de levantamento de análise de requisitos do sistema. O cronograma inclui ainda um sprint adicional no dia 18 de setembro para refinamento da apresentação dos artefatos para o cliente.
\vspace{12mm}

\newpage
\chapter{Conclusão}
    \vspace{-6mm}
    \hspace{4.5mm}
Portanto, a proposta de informatização do Mangaba Kitchen visa transformar como o restaurante gerencia suas operações diárias, abordando diretamente as dificuldades enfrentadas na coordenação entre a equipe e no controle de pedidos. A introdução de um sistema integrado que inclui interfaces específicas para gerentes, garçons e cozinheiros permitirá uma gestão mais eficiente dos pedidos, um controle aprimorado das estimativas de entrega e uma comunicação mais fluida entre as áreas. Com funcionalidades como cadastro e gerenciamento de funcionários, acompanhamento de pedidos e geração de relatórios financeiros, o sistema visa não apenas otimizar a operação interna, mas também melhorar a experiência do cliente, reduzindo atrasos e erros que impactam a satisfação dos consumidores.
\par
A viabilidade econômica e técnica do projeto está bem sustentada pela clareza dos requisitos e pela experiência da equipe de desenvolvimento, que possui conhecimento profundo das ferramentas e tecnologias necessárias. O cronograma estruturado utilizando o framework Scrum, com fases definidas e sprints regulares, assegura que cada etapa do desenvolvimento será cuidadosamente monitorada e ajustada conforme necessário. A combinação de uma estratégia bem planejada, a escolha adequada de dispositivos e a abordagem iterativa garantem que o sistema será implementado eficientemente, alinhando-se às necessidades operacionais do Mangaba Kitchen e promovendo uma operação mais ágil e eficiente.


% Formatação da bibliografia
% bibliographystyle{plain}
% \newpage
% \bibliography{referencias} % Assume que você tem um arquivo referencias.bib

\end{document}

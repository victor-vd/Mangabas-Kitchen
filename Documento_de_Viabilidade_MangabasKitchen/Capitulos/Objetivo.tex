\hspace{4.5mm}
O principal objetivo do sistema para o Mangaba Kitchen é otimizar a operação do restaurante, melhorando o controle sobre as estimativas de entrega dos pratos e o gerenciamento dos pedidos. Com a crescente complexidade na coordenação entre a equipe de garçons e a cozinha, é fundamental estabelecer um sistema eficiente que não só simplifique a comunicação interna, mas também eleve a qualidade do serviço prestado aos clientes. A implementação de um sistema informatizado visa resolver os problemas atuais e promover uma operação mais organizada e ágil.
\par
Para atingir esse objetivo, o sistema será projetado com várias funcionalidades específicas. Primeiramente, será criado um controle de acesso para funcionários, com logins e senhas individuais, assegurando a segurança e a responsabilidade na operação. Além disso, uma interface para a cozinha permitirá a gestão estruturada dos pedidos, incluindo a capacidade de adicionar pratos à “fila de espera” e ajustar o tempo de preparo conforme necessário. Os cozinheiros também terão acesso às receitas detalhadas dos pratos, o que ajudará a garantir a precisão na preparação e a reduzir erros. Paralelamente, os garçons utilizarão uma interface dedicada para registrar pedidos diretamente, o que facilitará a comunicação do salão e da cozinha.
\par
Além dos aspectos operacionais, o sistema também abordará questões relacionadas à transparência e responsabilização. Será implementado um mecanismo para gerar relatórios detalhados sobre a comissão dos garçons e registrar erros na preparação dos pratos, aplicando desconto financeiro em caso de falhas. Também será criado um sistema de crítica para garantir que os garçons verifiquem restrições alimentares dos clientes, com penalizações em caso de negligência. Esses objetivos visam promover um ambiente de trabalho mais eficiente, seguro e orientado à qualidade, resultando em um melhor atendimento ao cliente e uma operação interna mais fluida e bem organizada.